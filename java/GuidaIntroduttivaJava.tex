\documentclass{article}
\usepackage[utf8]{inputenc}

\title{Guida all'utilizzo di java}
\author{Eros Rossi}
\date{December 2018}

\begin{document}

\maketitle

\section{Installazione del Java JDK}
\noindent Per sviluppare qualsiasi programma Java la prima cosa di cui abbiamo bisogno è il Java Development Kit (JDK). 

Se non lo hai già installato sulla tua macchina (puoi verificarlo aprendo un terminale e provando a lanciare il comando \texttt{java}), allora procedi con l'installazione:

\begin{itemize}
\item Se hai Windows o macOS devi scaricare il Java JDK 8 dal sito web di Oracle https://www.oracle.com/technetwork/java/javase/downloads/jdk8-downloads-2133151.html per il tuo sistema operativo, quindi eseguire il file scaricato per installarlo. 

\item Se invece hai Linux, dovrai installare il pacchetto \texttt{openjdk-8-jdk} tramite il Software Center grafico o tramite il comando da terminale \texttt{sudo apt install openjdk-8-jdk}
\end{itemize}

\section{Installazione di BlueJ}
Per l'installazione di BlueJ vai a questo link: www.bluej.org/versions.html e scarica la versione 4.1.4 per il tuo sistema operativo.
\noindent A download terminato segui le istruzioni per installarlo sul tuo sistema operativo:

\begin{itemize}
    \item macOS: nella cartella troverai giá il file BlueJ.app e si puó semplicemente spostare nella cartella /Applicazioni di modo da vederlo in Lauchpad.
    
    \item Windows: ti verrá scaricato un installer, ti basterá lanciarlo e proseguire cliccando su ``avanti'' fino a che il programma sarà installato. Ad installazione terminata potrai avviare BlueJ dal menù Start.
    
    \item Linux: se hai Ubuntu, fai riferimento alla pagina: https://bluej.org/debian-ubuntu-oracle-java.html\#ubuntu-install. In sostanza dovrai installare il pacchetto .deb che puoi scaricare dal sito facendovi doppio click sopra o utilizzando il comando da terminal \texttt{dpkg -i}. 
\end{itemize}

\section{Creare un progetto}
\noindent Per iniziare a scrivere il primo codice in Java seleziona dal menú in alto Progetto $\to$ Nuovo Progetto, inserisci il nome che preferisci e dove salvare il progetto; verrá creata una cartella contenente diversi file, dovrai occuparti solamente del file con estensione ".bluej".

\noindent Premi ora su crea nuova classe per poter iniziare a scrivere il tuo codice dovrai solamente darle un nome; la classe verrá automaticamente inizializzata con un semplice esempio che adesso andremo ad analizzare; per aprire l'editor di testo basta semplicemente premere due volte sul file da modificare.

\section{Struttura del codice Java}

\noindent Ora iniziamo ad analizzare il codice del primo esempio.
\subsection{Sezione 1: Le librerie}
    \noindent La prima sezione del codice é la porzione di codice dove vengono dichiarate le librerie di Java per poter compilare il codice (maggiori informazioni riguardo alle librerie saranno fornite durante il corso).

\subsection{Sezione 2: La classe}
    \noindent In questa sezione viene dichiarata la classe e le viene assegnato un nome (In Java la classe pubblica ed il file devono avere lo stesso nome, presta attenzione alle maiuscole!) é qui dentro che ti concentrerai a scrivere il tuo codice dove possiamo dichiarare i metodi e le variabili globali (argomento trattato più avanti nelle lezioni). Ogni classe contiene dati e metodi, i quali possiamo considerarli l'analogo delle funzioni del Python.
    
\subsection{Sezione 3: Il Main}
    \noindent Il metodo main é quello che viene chiamato per primo alla messa in esecuzione del codice. Il main accetta un parametro args che contiene i parametri specificati dalla riga di comando.
    
\subsection{Sezione 4: I Metodi}
    \noindent I metodi sono delle funzioni che ci permettono di ottenere un codice più leggibile, più chiaro; nel codice allegato vedete una semplice implementazione dei metodi, essi necessitano di alcune variabili in ingresso (anche nessuna) ed una o nessuna variabile in uscita.

\section{Compilazione ed Esecuzione}
\noindent Una volta scritto il tuo codice in java, prima di poterlo eseguire, è necessario effettuare la compilazione dei files sorgente. Ricordati di salvarli con lo stesso nome della classe principale che contengono.

\subsection{Da linea di comando}
Per compilare ed eseguire un programma Java dal terminale, è necessario fare i seguenti passi:
\begin{itemize}
\item Salvare il programma in un file con estensione \textit{.java} che deve obbligatoriamente chiamarsi come la classe contenuta (puoi fare una prova con il file \textit{PrimoEsempio.java})

\item Spostarsi con il terminale nella directory dove si è salvato questo file

\item Per compilare il programma lanciare il comando \texttt{javac <nomefile>.java}, per esempio \texttt{javac PrimoEsempio.java}. Se il programma non contiene errori, viene compilato e viene prodotto un file bytecode \texttt{<nomefile>.class}

\item Per eseguire il programma lanciare \texttt{java <nomefile>}, per esempio \texttt{java PrimoEsempio}
\end{itemize}


\subsection{Tramite IDE BlueJ}
\noindent Per creare un progetto dal quale potremo compilare ed eseguire un programma tramite IDE BlueJ, è necessario fare i seguenti passi:
\begin{itemize}
\item Dopo aver aperto BlueJ clicca su \textit{Project} in alto a sinistra. 
\item Dal menu a tendina clicca su \textit{New project...} e dalla finestra che appare scegli dove il tuo nuovo progetto dovrà essere creato e digita il nome che dovrà avere nel campo \textit{Folder Name}.
\item Crea una nuova classe con il pulsante \textit{New Class...}. Poi digita il suo nome, per esempio, \textit{PrimoEsempio} e mantieni la scelta su \textit{Class}. Clicca \textit{Ok}.
\item Vedrai comparire un rettangolo con il nome che hai scelto. Fai doppio click su di esso per modificarlo dato che per ora contiene un programma di esempio. Completa la scrittura del file sorgente oppure copia e incolla il codice da un altro file.
\item Per compilare clicca sul pulsante \textit{Compile}. Se non ci sono problemi vedrai che le righe del rettangolo con il nome della tua classe spariranno. Puoi chiudere la finestra di modifica del file sorgente.
\item Se la classe contiene anche un metodo \texttt{void main(String args[])} possiamo lanciarlo cliccando con il tasto destro del mouse sul rettangolo della nostra classe per poi cliccare \textit{void main(String args[])} 
\item Mantieni la scelta su \texttt{\{\}} ovvero non vogliamo passare nessun input. Vedrai comparire una schermata che corrisponde a un terminale (nel caso non succedesse clicca su \textit{View} e poi \textit{Show Terminal}.
\item Se ci fossero degli input da inserire lo puoi fare dal terminale.
\end{itemize}
Dal terminale: cliccando su \textit{Options} puoi selezionare \textit{Clear screen at method call} per fare in modo che a ogni lancio di programma la storia di stampe a video venga resettata.

\end{document}

